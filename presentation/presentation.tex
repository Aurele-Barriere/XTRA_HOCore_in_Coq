\documentclass{beamer}
\usepackage[utf8]{inputenc}
\usepackage[T1]{fontenc}
\usepackage[french]{babel}
\usepackage{amsmath,amsfonts,amssymb}
\usepackage{graphicx}

\usetheme{Warsaw}


\title{HOCore en Coq}
\date{10 mai 2016}

\author{Aurèle Barrière, sur un article de Petar Maskimovic \& Alan Schmitt}

\begin{document}

\begin{frame}
\maketitle
\end{frame}

\begin{frame}{Des calculs}
\begin{block}{$\lambda$-calcul}
\end{block}
\begin{block}{$\pi$-calcul}
\end{block}
\end{frame}


\begin{frame}{HOCore}
Calcul de processus

\begin{itemize}
\item Turing Complet
\item Minimal
\item Ordre supérieur
\end{itemize}

\textbf{H}igher \textbf{O}rder \textbf{Core}
\end{frame}

\begin{frame}{Syntaxe}
\begin{block}{Catégories syntaxiques}
Variables $x$\\
Canaux $a$\\
Processus $P$
\end{block}

La grammaire d'un processus en HOCore
$$ P ::= a(x).P \mid \bar{a}\langle P\rangle \mid P\|P \mid x \mid 0 $$

\end{frame}

\begin{frame}{Sémantique}

\begin{alertblock}{}
$ P ::= a(x).P \mid \bar{a}\langle P\rangle \mid P\|P \mid x \mid 0 $
\end{alertblock}


$0$ ne fait rien.

$x$ variable.

$P\|Q$ exécution en parallèle. Associative et commutative. Permet communication

$\bar{a}\langle P\rangle$ émission de $P$ sur le canal $a$.

$a(x).P$ réception sur le canal $a$ pour $x$ dans $P$.

\end{frame}

\begin{frame}{Simplifications}
Émission et réception sur un même canal.

$\bar{a}\langle P\rangle\|a(x).Q \rightarrow [P/x]Q$

$\rightarrow$ : réduction.

Parallèlisme associatif et commutatif donc 
$\bar{a}\langle P\rangle\|\bar{b}\langle Q\rangle\|a(x).x \equiv  \bar{b}\langle Q\rangle\|\bar{a}\langle P\rangle\|a(x).x$ $\rightarrow$ $\bar{b}\langle Q\rangle\|P$
\end{frame}

\begin{frame}{Exemple : récursivité}
Processus $P$. On cherche $!P$ tel que $!P\rightarrow\ P\|!P$.

Alors $!P$ va répliquer indéfiniment $P$.


Soit $L = r(x).(x\|\bar{r}\langle x\rangle)$.

Soit $R = \bar{r}\langle P\|r(x).(x\|\bar{r}\langle x\rangle)\rangle$

Montrons que $!P = L\|R$ convient.

$L$ et $R$ communqiuent sur $r$. Après communication, $R$ se réduit donc en $0$.

Dans $L$, on remplace $x$ par le message émis par $R$.


On a donc $$!P\rightarrow P\|r(x).(x\|\bar{r}\langle x\rangle)\ \|\ \bar{r}\langle P\|r(x).(x\|\bar{r}\langle x\rangle)\rangle\| 0$$

Et donc $!P\rightarrow P\|!P$
\begin{alertblock}{}
 $$!P = (r(x).(x\|\bar{r}\langle x\rangle))\ \|\ \bar{r}\langle P\|r(x).(x\|\bar{r}\langle x\rangle)\rangle$$
\end{alertblock}
\end{frame}

\begin{frame}{Exemple : choix de processus}
\end{frame}

\begin{frame}{Équivalence de processus}
\end{frame}

\begin{frame}{Formalisation en Coq}
\end{frame}

\begin{frame}{Représentation canonique des noms}
\end{frame}

\begin{frame}{Systèmes de transitions étiquetées}
\end{frame}

\begin{frame}{Correction de preuves}
\end{frame}

\begin{frame}{Conclusion}
\end{frame}

\end{document}
