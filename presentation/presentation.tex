\documentclass{beamer}
\usepackage[utf8]{inputenc}
\usepackage[T1]{fontenc}
\usepackage[french]{babel}
\usepackage{amsmath,amsfonts,amssymb}
\usepackage{graphicx}


\usetheme{Warsaw}

\def\+{\oplus}
\title{HOCore en Coq}
\date{10 mai 2016}

\author[Aurèle Barrière]{Aurèle Barrière, sur un article de Petar Maskimovic \& Alan Schmitt}

\begin{document}

\begin{frame}
\maketitle
\end{frame}

\begin{frame}{Des calculs}
\begin{block}{$\lambda$-calcul}
$$\Lambda = x \mid \Lambda\Lambda \mid \lambda x.\Lambda$$
$x$ variable, $\Lambda$ fonction.
\end{block}
\begin{block}{$\pi$-calcul}
$$ P ::= a(x).P \mid \bar{a}\langle P\rangle \mid P\|P \mid 0 \mid \nu n P\mid !P $$
$a$ canal, $x$ variable, $P$ processus.
\end{block}
\end{frame}


\begin{frame}{HOCore}
Calcul de processus

\begin{itemize}
\item Turing Complet
\item Minimal
\item Ordre supérieur
\end{itemize}

\textbf{H}igher \textbf{O}rder \textbf{Core}
\end{frame}

\begin{frame}{Syntaxe}
\begin{block}{Catégories syntaxiques}
Variables $x$\\
Canaux $a$\\
Processus $P$
\end{block}

La grammaire d'un processus en HOCore
$$ P ::= a(x).P \mid \bar{a}\langle P\rangle \mid P\|P \mid x \mid 0 $$

\end{frame}

\begin{frame}{Sémantique}

\begin{alertblock}{}
$ P ::= a(x).P \mid \bar{a}\langle P\rangle \mid P\|P \mid x \mid 0 $
\end{alertblock}


$0$ ne fait rien.

$x$ variable.

$P\|Q$ exécution en parallèle. Associative et commutative. Permet communication

$\bar{a}\langle P\rangle$ émission de $P$ sur le canal $a$.

$a(x).P$ réception sur le canal $a$ pour $x$ dans $P$.

\end{frame}

\begin{frame}{Simplifications}
Émission et réception sur un même canal.

$\bar{a}\langle P\rangle\|a(x).Q \rightarrow [P/x]Q$

$\rightarrow$ : réduction.

Parallèlisme associatif et commutatif donc 
$\bar{a}\langle P\rangle\|\bar{b}\langle Q\rangle\|a(x).x \equiv  \bar{b}\langle Q\rangle\|\bar{a}\langle P\rangle\|a(x).x$ $\rightarrow$ $\bar{b}\langle Q\rangle\|P$
\end{frame}

\begin{frame}{Exemple : récursivité}
Processus $P$. On cherche $!P$ tel que $!P\rightarrow\ P\|!P$.

Alors $!P$ va répliquer indéfiniment $P$.


Soit $L = r(x).(x\|\bar{r}\langle x\rangle)$.

Soit $R = \bar{r}\langle P\|r(x).(x\|\bar{r}\langle x\rangle)\rangle$

Montrons que $!P = L\|R$ convient.

$L$ et $R$ communiquent sur $r$. Après communication, $R$ se réduit donc en $0$.

Dans $L$, on remplace $x$ par le message émis par $R$.


On a donc $$!P\rightarrow P\|r(x).(x\|\bar{r}\langle x\rangle)\ \|\ \bar{r}\langle P\|r(x).(x\|\bar{r}\langle x\rangle)\rangle\| 0$$

Et donc $!P\rightarrow P\|!P$
\begin{alertblock}{}
 $$!P = (r(x).(x\|\bar{r}\langle x\rangle))\ \|\ \bar{r}\langle P\|r(x).(x\|\bar{r}\langle x\rangle)\rangle$$
\end{alertblock}
\end{frame}

\begin{frame}{Exemple : choix de processus}
Choisir entre $P$ et $Q$.

On souhaite donc avoir le processus  $(a_1.P\+ a_2.Q)$ et les processus $\hat{a}_1$ et $\hat{a}_2$ tels que

$$(a_1.P\+ a_2.Q)\|\hat{a}_1\rightarrow\up{*}\ P$$
$$(a_1.P\+ a_2.Q)\|\hat{a}_2\rightarrow\up{*}\ Q$$

\begin{alertblock}{}
$$(a_1.P\+ a_2.Q) = \bar{a}_1\langle P\rangle\ \|\ \bar{a}_2\langle Q\rangle$$
$$\hat{a}_1 = a_1(X).(a_2(Y).(X))$$
$$\hat{a}_2 = a_1(X).(a_2(Y).(Y))$$
\end{alertblock}
\end{frame}

\begin{frame}{Complétude}
On a récursivité, ordre supérieur, conditions...

HOCore est Turing-Complet.

Encodage dans les machines de Minsky.
\end{frame}
 
\begin{frame}{Équivalence de processus}
\begin{center}
En HOCore, l'équivalence de processus est décidable.
\end{center}
Équivalence très faible. 

$P$ et $Q$ équivalents ssi \begin{itemize}
\item Si$P$ se réduit en $P'$, il existe $Q'$ équivalent à $P'$ tel que $Q$ se réduit en $Q'$. 
\item $P$ et $Q$ ont les mêmes \textit{observables} : ils émettent des messages sur les mêmes canaux.  
\item Pour tout contexte $C$ (processus avec un trou), le contexte complété avec $P$, $C[P]$, est équivalent à $C[Q]$.
\end{itemize}
\end{frame}

\begin{frame}{Formalisation en Coq}
Intérêt de la formalisation : trouver une méthode pour décider de l'équivalence.

Syntaxe : OK

Sémantique : alpha-conversion

\begin{exampleblock}{Alpha Conversion}
Variables liées qui jouent le même rôle.

Exemple : $a(x).a(y).x$ et $a(z).a(y).z$ pour $x$ et $z$.

Mais pas les processus $x$ et $y$ (libres)
\end{exampleblock}
\end{frame}

\begin{frame}{Représentation canonique des noms}

Idée : fonction qui calcule à chaque variable un indice indépendament du nom.

Pollack, Sato et Riciotti 
%j'expliquerais quand j'aurais compris :'(

\end{frame}

\begin{frame}{Systèmes de transitions étiquetées}

Analyser comportement processus : trouver les réductions.
Mais deux processus qui communiquent en parallèle ne sont pas toujours à côté :
$$\bar{a}\langle P\rangle\|R\|S\|T\|U\|a(x).Q \text{ se réduit en } 0\|R\|S\|T\|U\|[P/x]Q$$

On étiquette le comportement de chaque processus pour trouver les réductions.

\begin{block}{Règles de LTS}
$$\text{OUT } \bar{a}\langle P\rangle \overset{\bar{a}\langle P\rangle}{\longrightarrow} 0$$
$$\text{IN } a(x).Q \overset{a(P)}{\longrightarrow} [P/x]Q$$
$$\text{TAU1 Si } P\overset{\bar{a}\langle R\rangle}{\longrightarrow} P' \text{ et } Q\overset{a(R)}{\longrightarrow} Q' \text{ alors } P\|Q \overset{\tau}{\longrightarrow} P'\|Q'$$
\end{block}

\end{frame}

\begin{frame}{Correction de preuves}
Formalisation en Coq : preuves assistées.

De nombreuses preuves sur le calcul contenaient des erreurs : hypothèses fausses, raisonnement sur de mauvaises structures...

Formalisation : 4kloc\\
Preuves : 22kloc
\end{frame}

\begin{frame}{Conclusion}
Formaliser la syntaxe, puis la sémantique en permettant à l'assistant de preuve de faire des réductions.

Résoudre les problèmes (alpha-conversion) pour aboutir à de nouvelles méthodes de preuves.

Corriger les preuves existantes, en apprendre plus sur ce type de calculs.

Première formalisation d'un $\pi$-calcul d'ordre supérieur.

Il reste des preuves à traduire en Coq.
\end{frame}

\end{document}
