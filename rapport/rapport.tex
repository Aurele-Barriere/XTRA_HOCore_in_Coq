\documentclass[11pt]{article}
\usepackage[utf8]{inputenc}
\usepackage[T1]{fontenc}
\usepackage[french]{babel}
\usepackage{amsmath,amsfonts,amssymb}
\usepackage[nottoc, notlof, notlot]{tocbibind}
\usepackage{fullpage}
\usepackage{hyperref}
\usepackage{abstract}

\title{HOCore en Coq : résumé}
\date{22 février 2016}

\author{Aurèle Barrière, sur un article de Petar Maskimovic \& Alan Schmitt}
\def\pic{$\pi$-calcul }
\def\barbue{\simeq}
\def\+{\oplus}

\begin{document}
\nocite{*}
\maketitle
\begin{abstract}
  Ce document résume l'article \cite{hocoreincoq}. Il présente la formalisation en Coq (l'assistant de preuve) du calcul de processus HOCore. On y présente le calcul et des propriétés, puis on s'intéresse aux enjeux de la formalisation et de la réalisation de preuves en Coq. 
  \begin{description}
  \item[Mots-clés:] Formalisation; Calcul; Processus; Coq; HOCore
  \end{description}
\end{abstract}

\section{Introduction}
L'article résumé ici (\cite{hocoreincoq}) traite de la formalisation en Coq de HOCore, un calcul de processus (similaire au lambda-calcul). L'enjeu est de contourner les problèmes qui se posent lors de la formalisation d'un tel calcul, puis de construire des preuves sur ce calcul. Certaines de ces preuves ayant déjà été faites à la main avant dans l'article \cite{expressiveness}, on pourra ainsi comparer les deux méthodes.

On présentera donc une introduction à HOCore, accompagnée d'exemples et de propriétés, comme la décidabilité de l'équivalence de processus.
On présentera ensuite les problèmes qui se posent pour formaliser le calcul et ses réductions en Coq.
Enfin, on s'intéressera aux conséquences de ce travail.

\section{Introduction à HOCore}
HOCore est un calcul de processus, semblable au lambda-calcul, utilisé en particulier pour décrire des exécutions distribuées de processus. HOCore est minimal (la syntaxe et les règles de calcul sont très petites) et d'ordre supérieur (un processus peut communiquer un autre processus comme message). 


\subsection{Syntaxe}
Un processus, en HOCore, suit la grammaire suivante :
$$ P ::= a(x).P \mid \bar{a}\langle P\rangle \mid P\|P \mid x \mid 0 $$

On va distinguer 3 catégories syntaxiques : des canaux sur lesquels émettre et recevoir des messages (dans la grammaire ci-dessus, $a$ est un canal), des variables (remplacées lors de la lecture d'un message, $x$ dans la grammaire) et des processus ($P$ dans la grammaire).

$a(x).P$ correspond à la lecture d'un message sur le canal $a$ pour l'exécution de $P$, tandis que $\bar{a}\langle P\rangle$ correspond à l'émission du processus $P$ sur le canal $a$.

Pour deux processus $P$ et $Q$, le processus $P\|Q$ correspond à l'exécution en parallèle de $P$ et $Q$.


\subsection{Sémantique et simplifications}
$0$ est un processus qui ne fait rien.

En HOCore, l'exécution en parallèle est associative et commutative, et le processus $0$ en est l'élément neutre. Elle permet la communication entre processus sur les canaux.

En effet, lorsqu'un processus émet un autre processus ou une variable sur un canal $a$ et qu'en parallèle un autre processus lit sur le même canal $a$, alors on peut faire la réduction suivante :

$\bar{a}\langle P\rangle\|a(x).Q \rightarrow [P/x]Q$, qui signifie que les occurrences de $x$ dans $Q$ sont remplacées par $P$. On utilisera le symbole $\rightarrow$ pour dire qu'un processus se réduit en un autre.

Grâce à l'associativité et la commutativité de la parallèlisation, deux processus peuvent communiquer même s'ils ne sont pas à côté : $\bar{a}\langle P\rangle\|\bar{b}\langle Q\rangle\|a(x).x \equiv  \bar{b}\langle Q\rangle\|\bar{a}\langle P\rangle\|a(x).x$ qui se réduit en $\bar{b}\langle Q\rangle\|P$, où $\equiv$ représente l'égalité de 2 processus à commutativité et associativité de $\|$ près.

Parmi les variables, il faut donc distinguer celles qui sont dites \textit{libres} et celles dites \textit{liées}. 
Une variable est liée lorsqu'elle peut être changée par la lecture sur un canal. Une variable est donc liée quand elle est dans la portée d'une réception. Ainsi, dans $a(x).P$, les occurences de $x$ dans $P$ sont liées. On peut rapprocher ces notions à celles de variables \textit{globales} et \textit{locales}.
Dans l'exemple de réduction précédent, $x$ est donc une variable liée puisqu'elle est remplacée à la lecture sur le canal $a$. Par contre, dans le processus $P\|x$, $x$ est une variable libre : elle ne va pas être remplacée. 

Enfin, dans le processus $\bar{a}\langle P\rangle\|(a(x).Q)\|x$,  qui se réduit donc en $[P/x]Q\|x$, il faut distinguer $x$ dans $Q$ qui est une variable liée puisque remplacée par $P$, et $x$ exécuté en parallèle, libre. 



\section{Exemples de processus en HOCore}
\subsection{Récursivité}

On peut se demander s'il est possible de répliquer un processus. Et en effet, HOCore permet de décrire des exécutions infinies.

En se donnant un processus $P$, on cherche donc un processus $!P$ tel que $!P\rightarrow\ P\|!P$. On aura ainsi un processus qui se réduit en un processus $P$ en parallèle de lui-même, qui va donc une fois de plus créer une instance de $P$ et ainsi de suite.

On prend $$!P = (r(x).(x\|\bar{r}\langle x\rangle))\ \|\ \bar{r}\langle P\|r(x).(x\|\bar{r}\langle x\rangle)\rangle$$
Montrons que ce processus convient. On va noter $L = r(x).(x\|\bar{r}\langle x\rangle)$ et $R = \bar{r}\langle P\|r(x).(x\|\bar{r}\langle x\rangle)\rangle$, de telle sorte que $!P = L\|R$.

On remarque que $R$ émet un processus sur le canal $r$ et qu'en parallèle $L$ reçoit sur le même canal. Les deux processus communiquent donc. Alors pour $L$, $x$ sera remplacé par tout le processus émis par $R$. $R$, une fois le message émis, se réduira en $0$.

On a donc $$!P\rightarrow P\|r(x).(x\|\bar{r}\langle x\rangle)\ \|\ \bar{r}\langle P\|r(x).(x\|\bar{r}\langle x\rangle)\rangle\| 0$$
Et on constate qu'on a $!P\rightarrow P\|!P\|0$, et donc $!P\rightarrow P\|!P$ comme on le souhaitait, puisque $0$ est l'élément neutre pour la parallèlisation.

Il est donc possible, avec HOCore, de répliquer un processus.

\subsection{Choix de processus}
On aimerait également disposer d'un processus qui choisisse entre deux processus $P$ et $Q$ selon si il est exécuté en parallèle d'un processus $\hat{a_1}$ ou $\hat{a_2}$.

On souhaite donc avoir le processus $(a_1.P\+ a_2.Q)$ et les processus $\hat{a}_1$ et $\hat{a}_2$ tels que
$$(a_1.P\+ a_2.Q)\|\hat{a}_1\rightarrow\up{*}\ P$$
$$(a_1.P\+ a_2.Q)\|\hat{a}_2\rightarrow\up{*}\ Q$$

En HOCore, cela peut se faire ainsi :
$$(a_1.P\+ a_2.Q) = \bar{a}_1\langle P\rangle\ \|\ \bar{a}_2\langle Q\rangle$$
$$\hat{a}_1 = a_1(X).(a_2(Y).(X))$$
$$\hat{a}_2 = a_1(X).(a_2(Y).(Y))$$

Ainsi, si on a en parallèle les processus $(a_1.P\+ a_2.Q)$ et  $\hat{a}_1$, le premier va émettre $P$ et $Q$ sur les canaux $a_1$ et $a_2$, qui seront lus par le second, qui lui même ne va exécuter après lecture que le processus émis sur $a_1$ à savoir $P$. 

\subsection{Turing Complétude}
On a montré qu'on pouvait avoir des comportements infinis ou des choix. On peut aussi encoder, comme en lambda-calcul, des entiers ou faire un test à 0. %a montrer?
On a en particulier que HOCore est un calcul Turing-Complet.
Cela est montré dans l'article \cite{expressiveness} en encodant les machines de Minsky dans HOCore.

Le problème de la terminaison de processus est indécidable pour HOCore.


\section{Équivalence de processus}
En HOCore, l'équivalence de processus est décidable. Pour comprendre cela, il convient de définir ce que sont deux processus équivalents.

Deux processus $P$ et $Q$ sont dits équivalents si :
\begin{itemize}
\item S'il existe une réduction de $P$ à un processus $P'$, il existe $Q'$ équivalent à $P'$ tel qu'il y a une réduction de $Q$ à $Q'$. 
\item $P$ et $Q$ ont les mêmes \textit{observables} : ils émettent des messages sur les mêmes canaux.  
\item Pour tout contexte $C$ (un processus avec un trou), le contexte complété avec $P$, $C[P]$, est équivalent à $C[Q]$.
\end{itemize}
La relation est symétrique. Une des particularités de HOCore (contrairement au \pic par exemple), est que les processus ne peuvent pas \textit{cacher} des variables. On peut donc choisir un contexte qui explore le processus et ses actions sur les canaux et les variables. 
%SYMETRIQUE

%justifier le choix de Coq? c'est plus facile les équivalences en coq qu'avec Isabelle?

\paragraph{Remarque :} Cela peut paraître surprenant quand on sait qu'en général le problème d'équivalence de fonctions est indécidable. Cependant, il ne s'agit pas d'un paradoxe puisque c'est une équivalence de processus très fine : seuls des processus quasiment identiques sont équivalents. Si on avait choisi l'égalité syntaxique, on aurait également une équivalence décidable.  Par exemple, cela ne résout pas le problème de l'arrêt : on sait que si deux processus sont équivalents, soit les deux terminent, soit aucun ne termine mais on ne dispose pas d'un processus indiquant si un autre processus termine. 


\section{Formalisation en Coq}
Un des principaux travaux de l'équipe de recherche a été de formaliser HOCore en Coq (l'assistant de preuve).
La syntaxe d'HOCore se traduit simplement en Coq, cependant des problèmes subsistent lorsqu'on s'intéresse à la sémantique : il faut pouvoir reconnaître les variables liées dont le rôle est identique. En effet, deux processus peuvent s'écrire différemment mais être équivalents.

Par exemple, $a(x).a(y).x$ et $a(z).a(y).z$ sont équivalents, mais ne s'écrivent pas rigoureusement de la même manière.

\subsection{Alpha-conversion et représentation canonique locale des noms}
Dans l'exemple précédent, on parle d'alpha-conversion si deux processus ont des variables liées dont le nom a été changé. Cependant, il faut bien distingeur variables liées et variables libres. On ne peut pas parler d'alpha-conversion pour des variables libres : le contexte peut utiliser ces variables libres et on pourrait alors perdre l'équivalence.

Ainsi, le processus $x$ n'est pas équivalent au processus $y$. Si on les met dans un certain contexte, par exemple $C = y\|\bar{a}\langle z\rangle\|a(x).(\ )$, on obtient les processus suivant : 
$$C[x] = y\|\bar{a}\langle z\rangle\|a(x).(x) \rightarrow y\|0\|z \rightarrow y\|z$$
$$C[y] = y\|\bar{a}\langle z\rangle\|a(x).(y) \rightarrow y\|0\|y \rightarrow y\|y$$
qui ne sont pas équivalents.

Si on veut pouvoir décider, avec Coq par exemple, de l'équivalence de deux processus, cela peut être un problème.

Pour le contourner, on peut entièrement se dispenser des noms de variables : on va disposer d'une fonction qui pour chaque variable va calculer une hauteur de manière identique et indépendante du nom. Cette approche est celle de \textit{représentation canonique locale des noms} introduite dans l'article \cite{canonical} .

On n'a plus alors d'alpha-conversion possible : les variables ne sont plus que des indices, qui seront égaux dans le cas de processus équivalents.

Pour reprendre l'analogie avec les variables globales et locales d'un programme, on peut dire que deux fonctions dont on a juste changé le nom des variables locales sont équivalentes. Mais si la fonction utilise des variables globales et qu'on modifie leur nom, on perd l'équivalence puisque les fonctions n'auront pas le même comportement dans des programmes qui utilisent ces variables.


\subsection{Système de transitions étiquetées} 
Pour pouvoir analyser un processus, trouver les communications entre les différents processus en parallèles sans modifier complètement sa syntaxe, on introduit une nouvelle forme de sémantique : un système de transitions étiquetées ou LTS (\textit{labeled transition system}).

En effet, notre réduction pour la communication se formalise ainsi :
$$\bar{a}\langle P\rangle\|a(x).Q \rightarrow [P/x]Q$$

Cependant, on sait qu'on a défini la parallèlisation comme étant transitive. On peut donc se retrouver dans le cas
$$\bar{a}\langle P\rangle\|R\|S\|T\|U\|a(x).Q \text{ qui se réduit en } 0\|R\|S\|T\|U\|[P/x]Q.$$

Comment alors repérer la communication entre processus sans changer toute la syntaxe pour se retrouver avec le processus récepteur juste après le processus émetteur?

Le principe du LTS est d'indiquer le comportement de chaque processus et les réductions possibles avec celui-ci. Ainsi on va indiquer pour chaque processus émetteur ou récepteur le canal sur lequel on émet ou on reçoit, et quel processus est émis. Il sera alors simple de détecter les communications possibles : pour un nombre quelconque de processus en parallèle, on les exprime avec le système de transitions étiquetées, puis on regarde les étiquettes pour voir si une réduction est possible. On dispose également d'une étiquette $\tau$ pour indiquer une réduction interne (qui n'émet ni ne reçoit aucun processus).

On a par exemple les règles suivantes : 
$$\text{OUT } \bar{a}\langle P\rangle \overset{\bar{a}\langle P\rangle}{\longrightarrow} 0$$
$$\text{IN } a(x).Q \overset{a(P)}{\longrightarrow} [P/x]Q$$
$$\text{TAU1 Si } P\overset{\bar{a}\langle R\rangle}{\longrightarrow} P' \text{ et } \overset{a(R)}{\longrightarrow} Q' \text{ alors } P\|Q \overset{\tau}{\longrightarrow} P'\|Q'$$
%faut  que je réussisse à mettre les étiquettes au dessus des flèches

Sans ce système, le seul principe d'induction disponible pour faire des preuves est sur la syntaxe des processus. Ici, on va pouvoir se permettre de raisonner par induction sur les transitions.

Ce système permet alors de définir des techniques de preuves pour caractériser l'équivalence.

\subsection{Correction de preuves}
Un des avantages de formaliser avec Coq HOCore a été de repérer des fautes dans des démonstrations de l'article \cite{expressiveness}.

Par exemple, une des preuves raisonne inductivement sur des tailles de processus mais en utilisant une structure différente de HOCore.

Dans une autre preuve, on affirme implicitement que la décomposition première d'un processus en forme normale reste en forme normale alors qu'il y a des contre-exemples.

Certaines erreurs peuvent amener à redéfinir une notion pour pouvoir rester cohérent avec le reste des travaux.

Ces erreurs sont faciles à commettre à la main lorsque la complexité de la preuve en cache les subtilités, et refaire ces preuves en Coq garantit leur validité. Il s'agit cependant d'une grande partie du travail à effectuer : si la formalisation de Coq a nécessité 4000 lignes de code, les preuves s'étendent sur 22000 lignes. 

Le développement est disponible à l'adresse \url{www.irisa.fr/celtique/aschmitt/research/hocore/}.

\section{Conclusion}
Le travail réalisé a donc consisté à formaliser la syntaxe d'HOCore en Coq, puis sa sémantique en permettant à l'assistant de preuve d'effectuer des réductions. 
On doit alors utiliser des procédés comme par exemple l'approche canonique locale des noms pour reconnaître les variables locales identiques.
On arrive ainsi à avoir une procédure pour montrer l'équivalence de deux processus. %decidability procedure for IO-bisimulation 
Ensuite, de nombreuses preuves sur ce calcul ont été faites en Coq. La vérification de Coq a permis de détecter des erreurs dans des preuves faites à la main, ou des définitions mal posées. Ce travail a également amélioré les connaissances sur HOCore et corrigé les erreurs dans les preuves.

Il reste cependant des preuves à traduire en Coq et des résultats à formaliser. %exemples? je suis pas sur de les avoir bien compris

À la connaissance des auteurs, il s'agit de la première formalisation d'un \pic d'ordre supérieur. 
\newpage
\bibliographystyle{plain}
\bibliography{rapport.bib}

\end{document}
