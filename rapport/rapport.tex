\documentclass{article}
\usepackage[utf8]{inputenc}
\usepackage[T1]{fontenc}
\usepackage[french]{babel}
\usepackage{amsmath,amsfonts,amssymb}
\title{HOCore en Coq : résumé}


\author{Aurèle Barrière}
\def\pic{$\pi$-calcul }

\begin{document}
\maketitle
\tableofcontents

\section{Introduction à HOCore}
\subsection{Pi-calcul}
Le \pic est un langage formel utilisé pour décrire, en particulier, les éxécutions distribuées de processus. Sa syntaxe, très simple, décrit simplement l'éxécution en parallèle.

En \pic, on manipule des processus, qui peuvent s'éxécuter séquentiellement ou parallèlement et terminer ou non. Des canaux sont également disponibles pour la réception et l'émission de messages ou de variables.

Le \pic utilise donc la grammaire suivante :
\begin{align*}
P &= 0 & \text{fin du processus} \\
&| !P &\text{répéter le processus}\\
&| P||P &\text{lancer les deux processus en parallèle}\\
&| x(y).P &\text{lire un message sur le canal $x$ pour remplacer $y$, puis lancer $P$}\\
&| \bar{x}(y).P &\text{envoyer le message $y$ sur le canal $x$, puis lancer $P$}\\
&| (\nu x)P &\text{réserver le nom $x$ pour le processus $P$}\\
\end{align*}

Il s'agit d'un calcul Turing Complet.

\subsection{Pi-calcul d'ordre supérieur : HOPi}
Pour l'ordre supérieur, on se permet de communiquer par les canaux aussi bien des noms (variables) que des processus.

Dans la grammaire proposée plus haut, $x$ et $y$ peuvent donc désigner des processus.

\subsection{HOCore}
Il s'agit d'une restriction qui conserve le caractère Turing Complet du \pic d'ordre supérieur. 

La grammaire utilisée est la suivante :

\begin{align*}
P &= 0 \\
&| x \\
&| P||P\\
&| x(y).P\\
&| \bar{x}(P) \\
\end{align*}


\subsection{Réductions}



\section{Équivalence décidable}
\subsection{alpha-conversion}
\subsection{Indices de De Bruijn}
\section{Bissimilarités}



\section{Formalisation en Coq}
\subsection{axiomatisation}
\subsection{Correction de preuves}




\end{document}
