\documentclass{article}
\usepackage[utf8]{inputenc}
\usepackage[T1]{fontenc}
\usepackage[french]{babel}
\usepackage{amsmath,amsfonts,amssymb}
\title{HOCore en Coq : résumé}


\author{Aurèle Barrière}
\def\pic{$\pi$-calcul }

\begin{document}
\maketitle
\tableofcontents

\section{Introduction à HOCore}
\subsection{Pi-calcul}
Le \pic est un langage formel utilisé pour décrire, en particulier, les éxécutions distribuées de processus. Sa syntaxe, très simple, décrit simplement l'éxécution en parallèle.

En \pic, on manipule des processus, qui peuvent s'éxécuter séquentiellement ou parallèlement et terminer ou non. Des canaux sont également disponibles pour la réception et l'émission de messages ou de variables.

Le \pic utilise donc la grammaire suivante :
\begin{align*}
P &= 0 & \text{fin du processus} \\
&| !P &\text{répéter le processus}\\
&| P||P &\text{lancer les deux processus en parallèle}\\
&| x(y).P &\text{lire un message sur le canal $x$ pour remplacer $y$, puis lancer $P$}\\
&| \bar{x}(y).P &\text{envoyer le message $y$ sur le canal $x$, puis lancer $P$}\\
&| (\nu x)P &\text{réserver le nom $x$ pour le processus $P$}\\
\end{align*}

Il s'agit d'un calcul Turing Complet.

\subsection{Pi-calcul d'ordre supérieur : HOPi}
Pour l'ordre supérieur, on se permet de communiquer par les canaux aussi bien des noms (variables) que des processus.

Dans la grammaire proposée plus haut, $x$ et $y$ peuvent donc désigner des processus.

\subsection{HOCore}
Il s'agit d'une restriction qui conserve le caractère Turing Complet du \pic d'ordre supérieur. Il s'agit d'une restriction minimale. On peut le voir aussi comme du $\lambda$-calcul autorisant les calculs parallèles.

Les travaux de l'équipe se basent sur un premier arcticle : \textit{On the Expressiveness and Decidability of Higher-Order Process Calculi}, dans lequel est définie la syntaxe de HOCore. On y montre, entre autres, la Turing complétude.

La grammaire utilisée est la suivante :

\begin{align*}
P &= 0 \\
&| x \\
&| P||P\\
&| a(x).P \text{(à la lecture d'une variable $y$ sur $a$, toutes les instances de $x$ dans $P$ seront remplacées par des $y$)}\\
&| \bar{a}(P) \\
\end{align*}

On va distinguer 3 catégories : des canaux sur lesquels émettre et recevoir des messages, des variables (remplacées lors de la lecture d'un message) et des processus.

Parmi les variables, il faut distinguer celles qui sont dites \textit{libres} et celles dites \textit{liées}.
Une variable est liée lorsqu'elle peut être changée par la lecture sur un canal. 

En HOCore, on utilise un système de transitions labelées pour décrire l'éxécution des processus. On utilise soit une étiquette de la forme $\bar{a}(P)$ pour une émission, $a(P)$ pour une réception de processus, ou $\tau$ pour une transition interne : par exemple lorsqu'on a simultanément une émission et une réception sur un même canal.


\subsection{Exemples de processus en HOCore}
\paragraph{Exemple de substitution} $\bar{a}(P)||a(x).Q \rightarrow [P/x]Q$, qui signifie que les instances de $x$ dans $Q$ sont ramplacées par $P$.
\paragraph{Exemple de variables liées et libres} $a(x).(P||y)$. Ici, les occurences de $x$ dans $P$ sont liées alors que $y$ est libre. %exemple directement pris de l'article KAM





\subsection{Réductions}
Lorsqu'un processus attend un message sur un canal et qu'en parallèle, un autre processus émet un message sur ce même canal, on remplace toutes les instances de la variable.




\section{Équivalence décidable}
On peut montrer que le problème de décision de l'équivalence de 2 processus est décidable.

Cependant, le problème de terminaison reste indécidable.

On dit que deux processus sont équivalents si leur comportement est identique, quel que soit le contexte. On définit ainsi la \textit{congruence barue} : ...




\subsection{alpha-conversion}
Le nom donné aux variables n'importe pas dans la sémantique d'un processus, mais pose un problème pour l'équivalence de processus.

\subsection{Indices de De Bruijn}
Il existe un moyen de se débarrasser des noms de variables pour éviter tout problème lié à l'alpha réduction.
\section{Bissimilarités}



\section{Formalisation en Coq}
Un des principaux travaux de l'équipe de recherche a été de formaliser HOCore en Coq (l'assistant de preuve).

\subsection{Axiomatisation}
On traduit la grammaire de HOCore ainsi :

\subsection{Expression des transitions}
HOCore utilise un systeme de transition labelées (LTS). Pour la formalisation en Coq, cela pose le problme du nom des variables ``bound''.
\subsection{Correction de preuves}
Dans l'article [12], des preuves étaient fausses.




\end{document}
